%%%%%%%%%%%%%%%%%%%%%%%%%%%%%%%%%%%%%%%%%%%%%%%%%%%%%%%%%%%%%%%%%%%%%%%%%%%%%%%
%                      PAPER-SPECIFIC   DEFINITIONS
%%%%%%%%%%%%%%%%%%%%%%%%%%%%%%%%%%%%%%%%%%%%%%%%%%%%%%%%%%%%%%%%%%%%%%%%%%%%%%%

\def\app{\mbox{\sl@\/}}
\def\norm#1{\mbox{$#1\!\downarrow$}}
\def\nrm#1#2{\mbox{$#2\!\downarrow_{#1}$}}
\def\anrm#1{\nrm{\alpha}{#1}}
\def\bnrm#1{\nrm{\beta}{#1}}
\def\enrm#1{\nrm{\eta}{#1}}
\def\red#1{\mbox{$\,\succ_{#1}\,$}}
\def\ared{\red{\alpha}}
\def\bred{\red{\beta}}
\def\ered{\red{\eta}}
\def\con#1{\mbox{$\longrightarrow_{#1}$}}
\def\acon{\con{\alpha}}
\def\bcon{\con{\beta}}
\def\econ{\con{\eta}}
\def\str#1{\mbox{$\stackrel{\!\!\!*}{\longrightarrow_{#1}}$}}
\def\astr{\str{\alpha}}
\def\bstr{\str{\beta}}
\def\estr{\str{\eta}}
\def\eq{\mbox{$\approx$}}
\def\eqv#1{\mbox{$\approx_{#1}$}}
\def\aeqv{\eqv{\alpha}}
\def\beqv{\eqv{\beta}}
\def\eeqv{\eqv{\eta}}
\def\sym#1{\mbox{$\longleftrightarrow_{#1}$}}
\def\asym{\sym{\alpha}}
\def\bsym{\sym{\beta}}
\def\esym{\sym{\eta}}
\def\symstr#1{\mbox{$\stackrel{\!\!\!*}{\longleftrightarrow_{#1}}$}}
\def\dom#1{\mbox{\bf dom$(#1)$}}
\def\ran#1{\mbox{\bf ran$(#1)$}}
\def\var#1{\mbox{\bf var$(#1)$}}
\def\mgu#1{\mbox{\bf mgu$(#1)$}}
\def\res#1{\mbox{\bf res$(#1)$}}
\def\can#1{\mbox{\bf can$(#1)$}}
\def\ren#1{\mbox{$#1^{\sharp}$}}
\def\state#1#2#3#4{\mbox{$\langle#1\;|\;#2\;|\;#3\;|\;#4\rangle$}}
\def\empset{\mbox{\O}}
\def\pif{\mbox{\tt:-}}

\def\fig#1#2#3#4{
   \begin{figure}[#4]
     \vspace{#1}
     \hspace{-.3in}\special{picture #3 scaled 800}
     \caption{#2}
     \label{#3}
   \end{figure}
}

\def\pgm#1{{\tt #1}}
\def\code#1{{\em #1}}
\def\mode#1{{\it #1}}

%%%%%%%%%%%%%%%%%%%%%%%%%%%%%%%%%%%%%%%%%%%%%%%%%%%%%%%%%%%%%%%%%%%%%%%%%%%%%%%
%                         BEGINNING   OF   ARTICLE
%%%%%%%%%%%%%%%%%%%%%%%%%%%%%%%%%%%%%%%%%%%%%%%%%%%%%%%%%%%%%%%%%%%%%%%%%%%%%%%

\documentstyle{kluwer}

\begin{document}

\title{Performance of Two Common Lisp Programs \goodbreak On Several Systems}
\runningtitle{Performance of Two CL Programs On Several Systems}
\titlenote{This work was sponsored by the Biomedical Research Technology Program
of the National Institutes of Health under grant RR-00785 and by the Information
Systems Technologies office of the Defense Advanced Research Projects Agency
under contract N00039-86-C0033.}

\author{Richard Acuff}
\email{acuff@SUMEX-AIM.Stanford.EDU}

\address{Stanford University \\
Stanford, California  94305}

\startingpage{1}
\endingpage{36}
\volume{\it{draft}}
\pubyear{1989}

\preamble

\begin{abstract}

To assist in the evaluation of Lisp platforms for the Stanford University
Knowledge Systems Laboratory, 25 Common Lisp implementations were
benchmarked.  Run time and compilation time data on two moderate-sized
application programs are presented, along with data on the effect of compiler
optimization levels and on the impact of display I/O on run time.  For these
Lisp benchmarks, several systems did not rank where we expected them based on
speed ratings using other conventional measures.  Also, the rankings of
machines by Lisp speed differed for the two programs we tested.  The data
indicate that the performance of Lisp systems is very application dependent.
The quality of the software environment should play at least as strong a role in machine
selection as performance benchmarks.
\end{abstract}

\section{Introduction}


At Stanford University's Knowledge Systems Laboratory (KSL), a large amount
of software is written in Lisp.  Thus, the performance of Lisp systems is
often crucial to the productivity of the lab.  In order to assist us in
understanding the performance of different Lisp systems, we have undertaken
an informal survey of 25 Common Lisp implementations using two software
packages developed in the KSL.  The main goal of this survey was to
understand the execution speed performance of systems that we might use in
the KSL for research and development or dissemination of research results.
Secondary goals were to evaluate the effect of compiler optimizer settings on
execution speed and to evaluate the effect of reducing the amount of output
on execution speed.  

There have been a number of projects to measure the performance of Lisp
systems.  Gabriel's work~\cite{gabriel} is probably the best known, and is
the origin of the so-called ``Gabriel Benchmarks,'' a set of small test
programs for measuring specific aspects of Lisp system performance.  The
Gabriel benchmarks are extremely valuable to people trying to compare Lisp
systems, if used knowledgeably.  However, the aspects of a Lisp system
stressed by a particular program are often difficult to determine so that it
is usually best, where possible, to run a program on the systems in
question rather than attempting to dissect the program and forecast its
performance analytically.  Also, with the advent of numerous implementations
of Common Lisp~\cite{steele}, we can now use much larger test programs
without the bother and uncertainty of porting between dialects.

In this survey we have focused on execution speed which has long been an
important criterion for comparing computer systems.  The first comparison of
two systems solving the same problem (benchmarking) was probably made shortly
after the creation of the second computer, and benchmarking has been a
primary differentiator among computer systems ever since.  However, execution
speed benchmarks are only one aspect of the performance of systems,
especially Lisp systems.  Issues like programming and user environments,
compatibility with other systems, the ability to handle ``large'' problems, and
cost (hardware, software, and human) must also be considered, and, given a
machine that is ``fast enough,'' these other issues will almost always be the
overriding factor.

Descriptions of the programs used in this evaluation are given in
Section~\ref{testsoft}.  A description of the methodology used in performing
the tests is in Section~\ref{methodology}, and information about the Lisp
systems tested is in Section~\ref{systems}.  Data on the execution speed of
the test programs are presented in Section~\ref{runspeed}, followed by
compilation speed data and a comparison between compilation speed and
execution speed in Section~\ref{compspeed}.  The effect of choosing various
values for the \pgm{SPEED} and \pgm{SAFETY} options of the \pgm{OPTIMIZE}
declaration on the BB1 system are discussed in Section~\ref{optimize}.  The
effect of reducing the screen output of the SOAR benchmark is presented in
Section~\ref{output}.  Details of the test procedures and descriptions of the
systems tested are in the appendices.

\section{Test Software}
\label{testsoft}

The software systems used in these tests were SOAR~\cite{laird} and the BB1
blackboard core~\cite{hrI} and~\cite{hrII}.  These test programs
were chosen primarily because they are implemented in pure Common Lisp,
making them extremely portable\footnote{There were one or two small porting
difficulties that were traced to problems in the test code which had to be
fixed.  For instance, many systems allow \pgm{(INTERN "NAME" 'USER)} where others
require \pgm{(INTERN "NAME" (FIND-PACKAGE "USER"))}.  Also we were unable to get
SOAR to work in either versions 1.0 or 1.1 of Allegro Common Lisp for the
Mac�II due to unexplained software hangs so it is omitted from SOAR-related
charts.}.  Both are systems in daily use in the KSL but represent two
distinct research directions in terms of program function and structure.
These systems were initially developed in environments other than those
tested, and no attempt was made to optimize their performance for any of
these tests.  Neither of these systems is an intensive user of numeric
computation.

A copy of the Common Lisp source code used for these tests may be obtained
via anonymous FTP from \pgm{SUMEX-AIM.Stanford.EDU} in the \mbox{\pgm{89-02}}
directory.  Individuals with a serious interest in these sources but without
FTP access may contact the author by sending U.S.  Mail to ``Richard Acuff,
Stanford KSL, 701 Welch Road, Bldg.  C, Stanford, CA 94305'' or electronic
mail to ``\pgm{acuff@SUMEX-AIM.Stanford.EDU}''.

\subsection{SOAR}

SOAR is a heuristic-search based general problem solving architecture
developed by Paul Rosenbloom, {\it et.~al.} See~\cite{laird} for more
information on the SOAR system.

All test runs of SOAR were done solving an eight-puzzle problem in one of
three modes: mode \mode{A} (simply solve the problem), mode \mode{B} (solve
the problem while ``chunking'' or ``learning''), and mode \mode{C} (solve the
problem after having ``learned'' in mode \mode{B}).

An ``eight puzzle'' is a common children's game with 8 tiles, numbered 1 to 8,
on a 3 by 3 grid such that a tile adjacent to the empty place can be pushed
into it.  ``Solving the eight-puzzle problem'' consists of producing a series
of tile moves such that, from a given arbitrary starting configuration, the
eight puzzle ends up with all the tiles in numerical order, reading from the
upper left around the puzzle clockwise, with the empty place in the middle.

The version of SOAR used was 4.4.4, dated April 19, 1987.  It consists of 1
large LISP source file and 2 small SOAR files containing productions for
solving the eight-puzzle problem.  The LISP source is 10,661 lines (280,050
characters) of lightly commented code.

\subsection{BB1}

BB1 is a blackboard-based problem solving architecture developed by Barbara
Hayes-Roth.  For more information on the BB1 blackboard core, see~\cite{hrI}.
For further information on BB1, see~\cite{hrII}.  All references to BB1 in this
document refer only to the ``core'' blackboard parts of the system and do not
include any other layers of the problem solving architecture or the user
interface, as these components are not in pure Common Lisp.  All test runs of
BB1 went through three cycles of adding 10 items to the blackboard, accessing
those 10 items, and then deleting them.

The version of BB1 used was 1.2.  The LISP source used consists of 10 files
ranging from 36 lines (814 characters) to 3,396 lines (107,528 characters) of
lightly commented code, with a total of 8,722 lines (295,199 characters) of
code.

\section{Methodology}
\label{methodology}

All the tests were performed in as near to a ``normal'' working environment as
could be achieved.  We tried to duplicate the working conditions that a
researcher would likely have both in hardware and software.  Where possible
we selected test machines configured with the amount of memory, amount and
type of disk, type of display, etc.\ that a typical developer would purchase
and use.  We ran the software in a way that a developer using the system
would probably use it.  Thus, if it was normal to run with garbage collection
enabled, under a window system, within an editor, or in a multi-programming
environment, then that was done.  For instance, Sun machines were tested
under SunView with a couple of perfmeters running.  The HP machine was tested
while running in GnuEmacs on X.10.  MIT-style Lisp machines were run with all
networking and other background processing on, and no special process
priority.  No expert tuning or system configuration was done beyond what the
tester could do by reading over the user documentation.  All systems were
tested in single-user mode, which is the way those systems tested are normally used
for Lisp work.

We feel that although this methodology results in less repeatable and less
explainable results, it gives a good approximation to what the end user will
experience.  Where time allowed, multiple runs were made to ensure accurate
readings.  Unfortunately the collection of the raw data (ie.  arranging for
machine access and making the timed runs) proved to be an extremely time
consuming process, taking a day or more for some of the systems, so the
information in this report was collected over a long period of time (October,
1987 to August 1989) and some of the data may not reflect the performance of a system's
latest revision.

The procedures used for running the tests are fully described in
Appendix~\ref{testproc}.  The Common Lisp \pgm{TIME} macro was used to collect timing
information.  Most times were recorded to the nearest second.  When reported
by the \pgm{TIME} macro, some extra information, usually relating to paging,
memory management, ``kernel'' time, etc., were recorded, but are not analyzed
here.  If several runs were made, only the best number is reported herein for
the sake of brevity.  Wherever possible, source files were stored on local
disks (for the Sun 3/75 systems the files were on a Sun 3/180 NFS server on
the same subnet).

\section{Systems Under Test}
\label{systems}

The systems that we tested were chosen based on their availability to the
testers as well as their suspected usefulness in future KSL programming
efforts.  All of the systems tested were workstations, as we were not able to
obtain access to mainframe systems.  It is also the case that workstations,
with their bit-mapped displays and dedicated processors, currently provide
the best Lisp development environments, in our opinion, and thus were more
interesting to us.

A mnemonic code is used for each of the 25 systems.  Usually the code is the
model of the machine except where there is more than one Lisp for a machine
(as in the case of the Sun 3/75) in which case a letter is prefixed to
indicate the Lisp being used.  Table~\ref{map} gives a mapping between codes
and machine types.  See Appendix~\ref{descs} for detailed descriptions of
system configurations.

This report was initially released as KSL Technical Report 89-02 in January, 1989.  In August,
1989, three additional systems, \code{A-Sparc}, \code{F-Sparc}, and \code{NeXT}, were added
to this survey and the results for the Franz Extended Common Lisp on the Sun 3/75 were
updated with measurements made on the more recent Franz Allegro Common Lisp on the same
Sun 3/75 (\code{F-3/75}).  In June, 1989 some updated figures were added to the January, 1989
version of this paper.  The non-beta version of the Symbolics MacIvory
(\code{Maci}), representing an average of 18\% improvement over the beta
version, was included and the Symbolics 3600 running Release 6.1 was replaced
by a Symbolics 3650 running Genera 7.2 (\code{3650}).

\begin{table}[tb]
\begin{center}
\small{
\begin{tabular}{|l|r|l|} \hline
Code & Test Date & System Type\strut \\
\hline
\code{3/260}\strut & Summer 1988 & Sun 3/260 with Lucid Lisp\footnotemark \\
\code{3/60} & Summer 1988 & Sun 3/60 with Lucid Lisp \\
\code{386} & Spring 1988 & Compaq 386 with Lucid Lisp \\
\code{386T} & Spring 1988 & Compaq 386 portable with Lucid Lisp \\
\code{4/260} & Summer 1988 & Sun 4/260 with Lucid Lisp \\
\code{4/280} & Winter 1988 & Sun 4/280 with Lucid Lisp \\
\code{A-Sparc} & August 1989 & Sun SparcStation with \goodbreak Austin Kyoto CL \\
\code{DEC-II} & Fall 1987 & DEC MicroVax II with VaxLisp \\ 
\code{DEC-III} & Fall 1987 & DEC MicroVax III with VaxLisp \\
\code{Exp1} & November 1988 & Texas Instruments Explorer I \\
\code{Exp2} & November 1988 & Texas Instruments Explorer II \\
\code{Exp2+} & November 1988 & Texas Instruments Explorer II Plus \\
\code{F-3/75} & August 1989 & Sun 3/75 with Franz Allegro CL \\
\code{F-4/280} & January 1989 & Sun 4/280 with Franz Allegro CL \\
\code{F-Sparc} & August 1989 & Sun SparcStation with \goodbreak Franz Allegro CL \\
\code{HP} & Fall 1987 & Hewlett Packard 9000/350 \\
\code{K-3/75} & Fall 1987 & Sun 3/75 with Kyoto Common Lisp \\
\code{L-3/75} & Summer 1988 & Sun 3/75 with Lucid Lisp \\
\code{Mac2} & Spring 1988 & Apple Macintosh II with Allegro CL \\
\code{Maci} & April 1989 & Symbolics MacIvory \\
\code{mX} & November 1988 & Texas Instruments microExplorer \\
\code{NeXT} & August 1989 & NeXT with Franz Allegro Common Lisp \\
\code{RT} & Spring 1988 & IBM RT/APC with Lucid Lisp \\
\code{3650} & April 1989 & Symbolics 3650 \\
\code{XCL} & Winter 1988 & Xerox 1186 \\ \hline
\end{tabular}
}
\end{center}
\caption{Mapping between codes and system types}
\label{map}
\end{table}

\section{Execution Speed}
\label{runspeed}

Most of the tables and charts in this report refer to elapsed-times
(wall-clock time) in seconds.  Most of the tables and charts have the system
types ordered according to what seems to be the most interesting comparison.
We have attempted to group systems of allegedly comparable performance
(according to our perception formed from talking to vendor representatives,
talking to other users, reading reports, etc.)

It is worth noting that on almost all of the systems tested, virtual memory
paging was a negligible part of the overall run time for the tests.  Nor was
it a very significant factor during compilation, except in the case of
compiling SOAR on \code{A-Sparc}.  In general, we do not expect this to be true for
most application systems.  Indeed, we would not be surprised if paging time
were a major component of overall run time for most applications.

\subsection{BB1}
\footnotetext{The Lucid Common Lisp
products tested are versions prior to multi-programming
within the Lisp and prior to the inclusion of generation-based scavenging
garbage collection in those systems.  The Allegro Common Lisp was not tested
with multiprogramming enabled.}

The data for the run times of the BB1\footnote{These times are for default
settings of the \pgm{SPEED} and \pgm{SAFETY} optimization qualities discussed
in Section~\ref{optimize}.} tests are given in Table~\ref{bb1runt}.
Figure~\ref{bb1run} shows the data graphically.

\begin{table}[tb]
\begin{center}
\begin{tabular}{|l|r|c@{\hspace{.5in}}|l|r|} \cline{1-2} \cline{4-5}
Code & Sec && Code & Sec\strut \\
\cline{1-2} \cline{4-5} 
\code{Exp2}\strut & 27 && \code{NeXT} & 129 \\
\code{Exp2+} & 17 && \code{RT} & 75 \\
\code{4/260} & 56 && \code{DEC-III} & 63 \\
\code{4/280} & 34 && \code{Exp1} & 87 \\
\code{F-4/280} & 56 && \code{3/60} & 73 \\
\code{A-Sparc} & 58 && \code{L-3/75} & 90 \\
\code{F-Sparc} & 72 && \code{F-3/75} & 152 \\
\code{386} & 47 && \code{K-3/75} & 96 \\
\code{386T} & 54 && \code{HP} & 115 \\
\code{mX} & 33 && \code{DEC-II} & 207 \\
\code{Maci} & 103 && \code{XCL} & 559 \\
\code{3650} & 76 && \code{Mac2} & 254 \\
\code{3/260} & 62 && & \\
\cline{1-2} \cline{4-5}
\end{tabular}
\end{center}
\caption{Run times for BB1}
\label{bb1runt}
\end{table}

\fig {4.13in}{BB1 Run (sec)}{bb1run}{tbh}
                                                                      
Systems that are marketed as comparable generally came out close to each
other with the following notable exceptions: \begin{itemize} \item There was
a significant difference between the \code{4/280} and the \code{4/260}.  Even though
the \code{4/260} had more memory, similar disk, more tuning effort, and was
tried with several later versions of Lisp it was consistently slower than the
\code{4/280} tested earlier.  We are at a loss to explain this discrepancy.  It is
also worth noting that, except for VaxLisp, Lucid Lisp seemed the most
difficult to tailor to a particular machine when it was being installed.

\item The DEC machines seem to be poor at running Lisp even though
they are usually thought of as competitive when running FORTRAN or C.

\item The microExplorer (\code{mX}) did better than expected probably
because its weak point, paging, was not stressed by this test.

\item On the same Sun 3/75, Franz Lisp did surprising poorly relative
to Lucid Lisp.  We had expected them to be much closer in performance.

\item \code{XCL} was over twice as slow as the nearest competitor.
\end{itemize}

\subsection{SOAR}

The data for the SOAR run tests are given in Table~\ref{soarrunt} and presented
graphically in Figure~\ref{soarrun}.  The figures are for the sum of the \mode{A},
\mode{B}, and \mode{C} modes.\footnote{The \mode{A} and \mode{C} mode figures
are for the ``no trace'' configuration as described in Section~\ref{output}.}

\begin{table}[tb]
\begin{center}
\small
\begin{tabular}{|l|r|c@{\hspace{.5in}}|l|r|} \cline{1-2} \cline{4-5}
Code & Run Time && Code & Run Time\strut \\
\cline{1-2} \cline{4-5}
\code{Exp2}\strut & 94 && \code{NeXT} & 374 \\
\code{Exp2+} & 62 && \code{RT} & 177 \\
\code{4/260} & 58 && \code{DEC-III} & 454 \\
\code{4/280} & 82 && \code{Exp1} & 369 \\
\code{F-4/280} & 120 && \code{3/60} & 187 \\
\code{A-Sparc} & 190 && \code{L-3/75} & 278 \\
\code{F-Sparc} & 108 && \code{F-3/75} & 584 \\
\code{386} & 126 && \code{K-3/75} & 697 \\
\code{386T} & 151 && \code{HP} & 219 \\
\code{mX} & 154 && \code{DEC-II} & 1851 \\
\code{Maci} & 296 && \code{XCL} & 1519 \\
\code{3650} & 219 && \code{Mac2} & N/A \\
\code{3/260} & 154 && & \\
\cline{1-2} \cline{4-5}
\end{tabular}
\end{center}
\caption{Aggregate Run Times for SOAR (sec)}
\label{soarrunt}
\end{table}

\fig {4.13in}{Sum of SOAR run times (sec)}{soarrun}{tbh}

Once again most systems fit where expected with the following notes:
\begin{itemize}

\item The Lucid Sun 4's are somewhat faster than the TI Explorer II
for the SOAR test whereas the opposite was true for the BB1 test.

\item The \code{F-3/75} was slower (584 versus 484) than in the much older
product tested in previous versions of this survey.

\item \code{XCL} and \code{DEC-II} were over twice as slow as the nearest other
system.
\end{itemize}

\fig {6.5in}{Normalized Run Times (time/average-time)}{normalizedrun}{tbhp}

\subsection{Normalized Run Times}

A given system, call it $\alpha$, may have run the SOAR test faster than
another system, $\beta$, while $\beta$ was faster for BB1.
Figure~\ref{normalizedrun} depicts this difference.  For both BB1 and SOAR
the run times have been normalized by dividing the run time by the mean of
the run times for all the machines, leaving out \code{DEC-II}, \code{Mac2},
and \code{XCL} to improve readability.

Lucid Lisp seemed to perform relatively better with SOAR than with BB1 in all
cases, while VaxLisp and, to a much lesser extent, the dedicated Lisp
machines, seemed to do better with BB1.

There are many possible explanations for these variations, but trying to
analyze each of them was well beyond the scope of this study.  The reasons
are most likely a result of differences among implementations in the
efficiency of various operations, some of which are used by SOAR but not by
BB1 and vice versa.  For instance, SOAR might make heavy use of hashing while
BB1 might make heavy use of list primitives, or one system might include a large
number of \pgm{SETQ} operations while the other might be more applicative in
nature.  The developers of SOAR and BB1 do not currently have information on
the aspects of the Lisp systems stressed by their software.

\begin{table}[tb]
\begin{center}
\small
\begin{tabular}{|l|r|r|c@{\hspace{.5in}}|l|r|r|}
\cline{1-3} \cline{5-7}
Code & SOAR & BB1 && Code & SOAR & BB1\strut \\
\cline{1-3} \cline{5-7}
\code{Exp2}\strut & 132 & 89 && \code{NeXT} & 397 & 372 \\
\code{Exp2+} & 78 & 76 && \code{RT} & 574 & 586 \\
\code{4/260} & 307 & 324 && \code{DEC-III} & 423 & 633 \\
\code{4/280} & 523 & 482 && \code{Exp1} & 520 & 327 \\
\code{F-4/280} & 535 & 264 && \code{3/60} & 569 & 551 \\
\code{A-Sparc} & 30,068 & 1,113 && \code{L-3/75} & 1,040 & 919 \\
\code{F-Sparc} & 340 & 259 && \code{F-3/75} & 492 & 380 \\
\code{386} & 386 & 355 && \code{K-3/75} & 1,365 & 1,234 \\
\code{386T} & 479 & 416 && \code{HP} & 237 & 235 \\
\code{mX} & 152 & 186 && \code{DEC-II} & 1,227 & 1,774 \\
\code{Maci} & 757 & 766 && \code{XCL} & 1,800 & 1,927 \\
\code{3650} & 391 & 466 && \code{Mac2} & N/A & 349 \\
\code{3/260} & 687 & 540 && & &\\
\cline{1-3} \cline{5-7}
\end{tabular}
\end{center}
\caption{Compilation Times (sec)}
\label{comptimet}
\end{table}

\fig {6.5in}{Compilation Time (sec)}{comptime}{tbhp}

\section{Compilation Speed}
\label{compspeed}

Developers and researchers must worry about how fast their programs compile
as well as how fast they run.  SOAR and BB1 compilation times are given in
Table~\ref{comptimet} and Figure~\ref{comptime}.

Figures~\ref{relperf} compares run time with compile time.  The ratio of
compilation time to run time is shown.  A system with a high rating spends
relatively more time compiling than running.  The absolute value of these
numbers have little meaning.  They are only useful for comparing systems.
                                                                      
\fig {6.5in}{Relative Performance of Compiler (Compile-Time/Run-Time)}{relperf}{tbhp}

As one might expect, the specially microprogrammed Lisp machines had
relatively fast compilers.  Some machines with run times slower than
predicted spent relatively less time compiling.  For example, the VaxLisp
compiler was relatively fast, but generated very slow code.  The Lucid
compiler seemed to take a long time but generated fast code.  The Allegro
Common Lisp for the Mac II took little time but still somehow generated
impressively fast code for BB1.


The extremely long compile time for SOAR on \code{A-Sparc} (at 8 hours and 21
minutes nearly 17 times longer than the next nearest) is somewhat surprising.
Approximately 15 minutes were spent in the Lisp-to-C phase of compilation, 2
hours were spent in the C compiler, and the remaining approximately 6 hours
were spent in the Unix assembler.  We conjecture that the assembler was
thrashing since the disk access light was on almost continuously for the
duration of the assembly.  It is worth noting that we have compiled larger
files (1,200KB versus 280KB) with \pgm{(SPEED~0) (SAFETY~3)} in approximately
an hour and thus suspect that the high level of optimization undertaken by
default in AKCL contributes to this extraordinary compilation time.


\section{Effect of \pgm{OPTIMIZE} Settings on BB1}
\label{optimize}

\begin{table}[tb]
\begin{center}
\begin{tabular}{|l|r|r|r|r|} \hline
Code & Default & (3, 0) & (0, 3) & (3, 2)\strut \\
\hline
\code{Exp2} & 27 & {\bf 25} & 27 & 25\strut \\
\code{Exp2+} & 17 & {\bf 17} & 18 & 18 \\
\code{4/260} & 56 & {\bf 46} & 47 & 46 \\
\code{4/280} & 34 & {\bf 34} & 48 & 34 \\
\code{F-4/280} & 56 & 56 & 56 & {\bf 54} \\
\code{A-Sparc} & 58 & {\bf 58} & 94 & 64 \\
\code{F-Sparc} & 72 & 69 & 69 & {\bf 68} \\
\code{386} & 47 & {\bf 47} & 52 & 47 \\
\code{386T} & 54 & {\bf 54} & 60 & 54 \\
\code{mX} & {\bf 33} & 34 & 34 & 30 \\
\code{Maci} & 103 & {\bf 100} & 100 & 104 \\
\code{3650} & 76 & {\bf 74} & 74 & 75 \\
\code{NeXT} & 129 & {\bf 126} & 131 & 130 \\
\code{3/260} & 62 & {\bf 62} & 69 & 62 \\
\code{RT} & 75 & 76 & 77 & {\bf 75} \\
\code{DEC-III} & 63 & {\bf 60} & 71 & 70 \\
\code{Exp1} & 87 & 87 & 90 & {\bf 83} \\
\code{3/60} & 73 & {\bf 72} & 76 & 72 \\
\code{L-3/75} & 90 & {\bf 90} & 127 & 90 \\
\code{F-3/75} & 152 & {\bf 150} & 152 & 152 \\
\code{K-3/75} & 96 & 165 & 147 & {\bf 88} \\
\code{HP} & 115 & {\bf 113} & 141 & 118 \\
\code{DEC-II} & 207 & {\bf 206} & 231 & 236 \\
\code{XCL} & 559 & {\bf 543} & 559 & 556 \\
\code{Mac2} & {\bf 254} & 258 & 261 & 259 \\
\hline
\end{tabular}
\end{center}
\caption{BB1 Run Times for Various \pgm{OPTIMIZE} Settings} 
\label{bb1runst}
\end{table}

\fig {6.5in}{BB1 runs with various \pgm{OPTIMIZE} settings (sec)}{bb1runs}{p}
\fig {6.5in}{BB1 compilation times with various \pgm{OPTIMIZE} settings (sec)}{bb1comps}{p}
\fig {4.125in}{BB1 Speedup Factors Due to \pgm{OPTIMIZE} Settings}{bb1speedup}{tbh}

The \pgm{OPTIMIZE} declaration is a way of controlling the behavior of a
Common Lisp compiler.  Two of the most significant qualities thus controlled
are \pgm{SPEED} and \pgm{SAFETY}.  Each of these can be set to an integer
from 0 to 3.  A high setting for \pgm{SPEED} tells the compiler that fast
running code is desired, which typically enables various optimizations.  The
Common Lisp specification doesn't require any optimizations or even that they
necessarily be controlled by this setting, but many current implementations
switch on optimizers such as dead code eliminators, tail and mutual recursion
eliminators, fancy register allocators, and facilities to take advantage of
type declarations.  The \pgm{SAFETY} quality is somewhat less well
understood.  It has little to do with the ``safety'' of the program since a
correct Common Lisp program is still required to run correctly if
\pgm{SAFETY} is low, but it has an impact on the debuggibility of the
program.  A high \pgm{SPEED} and low \pgm{SAFETY} may allow, for instance,
disabling number-of-arguments checking to allow faster function calls on some
architectures, or type checking on system functions (such as \pgm{CAR} or
\pgm{SETQ}) might be disabled.  Kyoto Common Lisp (KCL) goes so far as to
``hardwire'' function calls such that if \pgm{FOO} calls \pgm{BAR} and
\pgm{FOO} is compiled then if \pgm{BAR} is later redefined and \pgm{FOO}
isn't, \pgm{FOO} will continue to call the old version of \pgm{BAR}, thereby
destroying much of the flexibility of the Lisp.

We chose 4 settings of \pgm{SPEED} and \pgm{SAFETY} to study:
\begin{enumerate}

\item The default setting that the Lisp system has when it is initialized.
This is what most people use.

\item \pgm{(SPEED 3) (SAFETY 0)} written (3, 0) below, which should generate the
fastest code.

\item \pgm{(SPEED 0) (SAFETY 3)} written (0, 3) below, which should generate slow
but very debuggible code, since the compiler should have done very few, if
any, optimizations.

\item \pgm{(SPEED 3) (SAFETY 2)} written (3, 2) below, which should generate
optimized code while retaining ``sanity checks''.

\end{enumerate}

The BB1 system used in these tests has very few declarations and does little
numerical work.  Both of these attributes seem common among most Common Lisp
programs we use.

Table~\ref{bb1runst} and Figure~\ref{bb1runs} give the results for running
BB1 with the four \pgm{OPTIMIZE} settings.  Figure~\ref{bb1comps} shows the
compilation times for the various \pgm{OPTIMIZE} settings.

\begin{sloppypar}
These charts reveal somewhat surprising results.  In several cases,
\pgm{(SPEED~3) (SAFETY~0)} did not give the best results!  Lucid Lisp did
consistently better when \pgm{SPEED} was higher than \pgm{SAFETY}, as did the
HP 9000, and VaxLisp.  KCL was definitely behaving strangely with \pgm{(SPEED~0)
(SAFETY~3)} coming out a good bit faster than \pgm{(SPEED~3) (SAFETY~0)}, with
both of those much slower than ``default'' or \pgm{(SPEED~3) (SAFETY~2)}.
\end{sloppypar}

Figure~\ref{bb1speedup} depicts the speedup factor between the slowest time
and the fastest time for the BB1 tests with various \pgm{OPTIMIZE} settings.
                                                                      

\section{Effect of Output Reduction on SOAR}
\label{output}

\begin{table}[tb]
\begin{center}
\begin{tabular}{|l|r|r||r|r|} \hline
&\multicolumn{2}{|c||}{Mode A\strut}&\multicolumn{2}{|c|}{Mode B} \\ \cline{2-5}
Code & Full & Reduced & Full & Reduced\strut\\ \hline
\code{Exp2}\strut & 33 & 18 & 18 & 16 \\
\code{Exp2+} & 23 & 11 & 13 & 11 \\
\code{4/260} & 23 & 13 & 11 & 9 \\
\code{4/280} & 35 & 15 & 14 & 11 \\
\code{F-4/280} & 36 & 36 & 20 & 19 \\
\code{A-Sparc} & 104 & 31 & 34 & 19 \\
\code{F-Sparc} & 58 & 23 & 18 & 14 \\
\code{386} & 41 & 27 & 21 & 19 \\
\code{386T} & 52 & 31 & 26 & 23 \\
\code{mX} & 50 & 27 & 29 & 27 \\
\code{Maci} & 107 & 79 & 45 & 35 \\
\code{3650} & 81 & 41 & 41 & 32 \\
\code{NeXT} & 97 & 82 & 68 & 64 \\
\code{3/260} & 49 & 33 & 23 & 22 \\
\code{RT} & 61 & 36 & 32 & 28 \\
\code{DEC-III} & 95 & 76 & 95 & 92 \\
\code{Exp1} & 90 & 63 & 75 & 71 \\
\code{3/60} & 66 & 38 & 34 & 31 \\
\code{L-3/75} & 82 & 67 & 45 & 41 \\
\code{F-3/75} & 152 & 135 & 106 & 100 \\
\code{K-3/75} & 186 & 136 & 120 & 111 \\
\code{HP} & 61 & 51 & 52 & 52 \\
\code{DEC-II} & 351 & 283 & 390 & 401 \\
\code{XCL} & 473 & 390 & 243 & 232 \\
\hline
\end{tabular}
\end{center}
\caption{SOAR Run Times with Full and Reduced Output}
\label{modeac}
\end{table}

%\fig {3.875in}{SOAR Mode A run time (sec)}{soara}{tbh}
%\fig {3.875in}{SOAR Mode C run time (sec)}{soarc}{tbh}
\fig {3.875in}{SOAR Speedup Due to Reduced Output}{soarspeedup}{tbh}

\begin{sloppypar}
The eight-puzzle benchmark for SOAR was originally written when SOAR ran
primarily on slower machines than those tested here.  Thus it tends to
generate a lot of output relative to the amount of computation for some of
the modes.  For some systems, particularly those with large bit-mapped
displays and full-screen windows, this output can be very expensive.  To
understand the extent of this effect we tested SOAR in the \mode{A} mode and
in the \mode{C} mode both with full output, and with greatly reduced output
(no trace).  Table~\ref{modeac} shows
results of these runs.  Figure~\ref{soarspeedup} depicts the amount of
speedup (ratio of run times) realized by SOAR with reduced output.
\end {sloppypar}

Three factors seemed to influence the speedup with reduced output:
\begin{itemize}
\item A fast processor, since the amount of time spent computing versus doing I/O would be 
reduced, causing a reduction in I/O time to be more significant.
\item A larger screen or window since it is expensive to scroll a large area.
\item A large-overhead I/O system such as the MacIvory's Dynamic Windows.
\end{itemize}
\section{Future Work}

Obvious areas in which this work might be extended include:
\begin{itemize}
\item Updating the results to reflect more recent versions of the
Common Lisp systems;
\item Adding more test systems, especially mainframes;
\item Benchmarking other programs besides SOAR and BB1;
\item Evaluating the effect of declarations on run times;
\item Adding measurements of storage management overhead;
\item Collecting more data on I/O overhead;
\item Understanding better why platforms vary in performance from
application to application and Lisp implementation to Lisp implementation.
\end{itemize}

\section{Conclusions}

Two moderate-sized applications, SOAR and BB1, were benchmarked on 25 Common
Lisp systems to help in the evaluation of different Common Lisp systems.  The
run and compile times for these benchmarks were presented and discussed.  A
large variation was observed between the ranking of systems when running the
SOAR test versus the ranking when running the BB1 test.  This leads us to
conclude that while these experimental results and ones like them can be used
to class machines together roughly, it is impossible to use such a set of
benchmarks to decide in advance how a given application will perform on a
given system.  There is no substitute for actually running the program on the
systems in question.

Figure 12 shows the average of the normalized\footnote{The data were normalized
by dividing each by the average of the results for all the tested
implementations.} run times for the test programs with the systems ranked in
order.  On the basis of this data, the systems tested may be ranked as follows:

\begin{itemize}
\item{Very Fast} ($\leq 0.50$ anr--averaged normalized run time):\hfill\break\quad
\begin{tabular}{ll}
\code{Exp2+} & TI Explorer II Plus\\
\code{Exp2} & TI Explorer II\\
\code{4/280} & Sun 4 with Lucid Lisp\\ 
\code{4/260} & Sun 4 with Lucid Lisp
\end{tabular}

\item{Fast} ($> 0.50$ anr, $< 1.00$ anr):\hfill\break\quad
\begin{tabular}{ll}
\code{mX} & TI microExplorer\\
\code{386} & Compaq 386\\
\code{F-4/280} & Sun 4 with Franz Lisp\\
\code{386T} & Compaq 386 portable\\
\code{F-Sparc} & Sun SparcStation with Franz Lisp\\
\code{3/260} & Sun 3/260\\
\code{A-Sparc} & Sun SparcStation with Austin KCL\\
\code{RT} & IBM RT/APC\\
\code{3/60} & Sun 3/60\\
\code{3650} & Symbolics 3650
\end{tabular}

\item{Medium} ($\geq 1.00$ anr, $\leq 1.50$ anr):\hfill\break\quad
\begin{tabular}{ll}
\code{L-3/75} & Sun 3/75 with Lucid Lisp\\
\code{HP} & HP 9000/350\\
\code{Maci} & Symbolics MacIvory\\
\code{Exp1} & TI Explorer I\\
\code{DEC-III} & DEC MicroVax III
\end{tabular}

\item{Slow} ($> 1.50$ anr, $\leq 2.50$ anr):\hfill\break\quad
\begin{tabular}{ll}
\code{NeXT} & NeXT 0.9 with Franz Allegro Common Lisp\\
\code{K-3/75} & Sun 3/75 with Kyoto Common Lisp\\
\code{F-3/75} & Sun 3/75 with Franz Allegro Common Lisp
\end{tabular}

\item{Very Slow} ($> 2.50$ anr):\hfill\break\quad
\begin{tabular}{ll}
\code{Mac2} & Apple Macintosh II with Allegro Lisp\\
\code{DEC-II} & DEC MicroVax II\\
\code{XCL} & Xerox 1186
\end{tabular}
\end{itemize}

\fig {4.125in}{Averaged Normalized Run Times}{anr}{tbh}

We were surprised at the high speed of the small 386 machines, and at the
slowness of the MacIvory, the DEC machines, and the Xerox machine.

Dedicated Lisp machines compile relatively faster than conventional machines,
and, generally, conventional machine systems that took more time to compile
produced faster code, as one would expect.

While the experiment to measure the effect of different settings of the
\pgm{OPTIMZE} declaration was interesting, with such a small sample no real
conclusion about the effect of various \pgm{OPTIMIZE} settings can be drawn.
However the indications are that, in the absence of other declarations (eg.
for \pgm{TYPE}), only relatively small gains are available.  It is probably
best to experiment with various settings to see which gets the best speed for
a given program.

Reducing the amount of output that a program generates can have a large
effect on the run time of the program, especially when moving the program to
a faster machine.  This indicates that it is worth taking some time to
consider the nature of the I/O system and interaction needed by a program
when designing a user interface for a fast-running program.

These results must be used very carefully since they represent only one piece
of information about the performance of the very complex systems tested.  We
have measured only execution speed, but many aspects of the software will
impact the development of programs such that in a given amount of time a
program might be written for one machine that runs faster and perhaps with
fewer errors than a program written in the same amount of time on another
machine that ranks faster in these tests due to superior support given to the
programmer during development.  Do not underestimate the power of the
programming environment.

\section{Acknowledgements}

This work would have been completely impossible without the assistance of
many people and companies.  Mike Kramer of Texas Instruments Inc.\ supplied
the Explorer II Plus processor board.  Eric Warner and Michael Borke of Sun
Microsystems Inc.\ supplied access to the Sun 4 systems and the Sun 3/260 and
3/60 systems.  Franz Inc.\ supplied a test version of Allegro Common Lisp for
the Sun 3.  Marty Hollander of Franz Inc.  supplied a version of Allegro
Common Lisp for the Sun 4.  Jeff Harvey of Digital Equipment Corp.\ arranged
access to the MicroVax systems.  Susan Rosenbaum and Eric Gilbert of Lucid
Inc.\ supplied access to the Compaq machines and the IBM RT.\@  Bruce Hamilton
of Hewlett Packard Inc.\ arranged access to the HP 9000.  Carl White of
Symbolics Inc.\ arranged access to the MacIvory and 3650.  Many thanks to all
of them.

\bibliographystyle{kluwer}

\begin{thebibliography}{10}

\bibitem{gabriel}
Gabriel, R.~P.\@
\newblock {\sl Performance and Evaluation of Lisp Programs}.
\newblock M.I.T. Press, Cambridge, Massachusetts (1985).

\bibitem{hrI}
Hayes-Roth, B.\@
\newblock A Blackboard Architecture for Control.
\newblock {\sl Artificial Intelligence Journal}, 26 (1985) 251--321.

\bibitem{hrII}
Hayes-Roth, B., and Hewett, M.\@
\newblock BB1:  An Implementation of the Blackboard Control Architecture.
\newblock In Engelmore, R. and Morgan, T., editors, {\sl Blackboard Systems},
Addison-Wesley (1988) 297--313.

\bibitem{laird}
Laird, J.~E., Newell, A., and Rosenbloom, P.~S.\@
\newblock Soar: An Architecture for General Intelligence.
\newblock {\sl Artificial Intelligence}, 33 (1987) 1--64.

\bibitem{steele}
Steele, G.~L.~jr.\@
\newblock {\sl Common Lisp the Language}.
\newblock Digital Press, Burlington, MA (1984).

\end{thebibliography}

\appendix

\section{System Descriptions}
\label{descs}

This appendix contains detailed descriptions of the systems used in these
measurements.  In the descriptions, the ``code'' used to refer to the systems
under test in this report is in the box above the configuration information
for a given system.  Usually the code is the model of the machine except where
there is more than one Lisp for a machine (as in the case of the Sun 3/75) in
which case a letter is prefixed to indicate the Lisp being used.  ``Timing
Template'' indicates how the information reported by the \pgm{TIME} macro was
recorded.  {\em Elapsed} indicates the total elapsed time, {\em run} indicates
CPU time used, {\em gc} indicates time spent in garbage collection, {\em user}
and {\em system} distinguish between user mode and kernel mode time, and
{\em paging} indicates time waiting for virtual memory disk operations.

\def\machdesca#1#2#3#4#5#6{
\begin{center}
\begin{tabular}{|r|p{2.9in}|}
\hline
\multicolumn{2}{|c|}{\bf #1\strut} \\ \hline
Computer Type: \strut & #2 \\
Operating System: & #3 \\
Lisp: & #4 \\
Disk Configuration: & #5 \\
Swapping Size: & #6 \\
Memory Configuration: }

\def\machdescb#1#2#3#4#5#6{& #1 \\
Display Configuration: & #2 \\
Other Configuration: & #3\\
Special Comments: & #4 \\
Timing Template: & #5 \\
Date-of-test: & #6 \\
\hline
\end{tabular}
\end{center}
}

\machdesca{3/260}
{Sun 3/260}
{Sun OS 3.4}
{Lucid 2.0}
{280MB}
{60MB}
\machdescb{8MB}
{Color in mono mode}
{}
{used \pgm{:EXPAND 130 :GROWTH-RATE 130}}
{elapsed (user-run + system-run)}
{Summer 1988}
\machdesca{3/60}
{Sun 3/60}
{Sun OS 3.4}
{Lucid 2.1}
{SCSI 141MB}
{unknown}
\machdescb{24MB}
{Hi Res Color in mono mode}
{}
{}
{elapsed (user-run + system-run)}
{Summer 1988}
\machdesca{3650}
{Symbolics 3650}
{Symbolics Release 7.2, ECO level 6}
{Symbolics Release 7.2, ECO level 6}
{unknown}
{200MB}
\machdescb{6,144kw (27MB)}
{}
{no color}
{EGC on}
{elapsed - paging}
{April 24, 1989}
\machdesca{386}
{Compaq 386 (20Mhz 386)}
{386/IX 5.3 rev level 1.01 (unix)}
{Lucid 2.0}
{134MB ESDI}
{unknown}
\machdescb{10MB; 32kB 20ns cache}
{terminal}
{}
{}
{elapsed (run)}
{Spring 1988}
\machdesca{386T}
{Compaq 386 portable (Toaster)}
{386/IX 5.3 rev level 1.01 (unix)}
{Lucid 2.0}
{40MB}
{unknown}
\machdescb{10MB; no cache}
{tiny LCD}
{tiny display}
{portable version of \code{386} above}
{elapsed (run)}
{Spring 1988}
\machdesca{4/260}
{Sun 4/260}
{SunOS 3.2 Gamma}
{Lucid 2.1}
{unknown}
{unknown}
\machdescb{32MB}
{Hi Res color in mono}
{}
{used \pgm{:EXPAND 130 :GROWTH-RATE 130}}
{elapsed (user-run + system-run)}
{Summer 1988}
\machdesca{4/280}
{Sun 4/280}
{SunOS 3.2 Gamma}
{Lucid 2.1 beta}
{417 (Eagle)}
{60MB}
\machdescb{8MB}
{Hi Res mono}
{}
{}
{elapsed (user-run + system-run)}
{Winter 1988}
\machdesca{A-Sparc}
{Sun SparcStation}
{Sun OS 4.0}
{Austin Kyoto Common Lisp}
{1x600MB SCSI}
{Large}
\machdescb{16MB}
{Color}
{}
{}
{elapsed (lispcpu)}
{August 1989}
\machdesca{DEC-II}
{DEC MicroVax II/GPX}
{VMS}
{VaxLisp}
{2 x 159MB}
{3k pg page, 8k pg swap}
\machdescb{16MB}
{GPX}
{}
{}
{elapsed - gc-elapsed (run - gc-run)}
{Fall 1987}
\machdesca{DEC-III}
{DEC MicroVax III (3500)}
{VMS}
{VaxLisp}
{(RD53)}
{unknown}
\machdescb{16MB}
{mono}
{}
{}
{elapsed - gc-elapsed (run - gc-run)}
{Fall 1987}
\machdesca{Exp1}
{Texas Instruments Explorer I}
{Explorer Lisp Release 4.1}
{Explorer Lisp Release 4.1}
{2 x 140MB SCSI}
{80MB}
\machdescb{8MB}
{1024 x 768 mono}
{}
{TGC on }
{elapsed - paging}
{November 1988}
\machdesca{Exp2}
{Texas Instruments Explorer II}
{Explorer Lisp Release 4.1}
{Explorer Lisp Release 4.1}
{2 x 140MB SCSI}
{80MB}
\machdescb{16MB}
{1024 x 768 mono}
{}
{TGC on}
{elapsed - paging}
{November 1988}
\machdesca{Exp2+}
{Texas Instruments Explorer II Plus}
{Explorer Lisp Release 4.1}
{Explorer Lisp Release 4.1}
{2 x 140MB SCSI}
{80MB}
\machdescb{16MB}
{1024 x 768 mono}
{}
{TGC on}
{elapsed - paging}
{November 1988}
\machdesca{F-3/75}
{Sun 3/75}
{SunOS 4.0}
{Franz Allegro Common Lisp 3.0.3}
{70MB SCSI}
{50MB local}
\machdescb{28MB}
{standard resolution mono}
{Files on Sun 3/180 NFS server}
{Under suntools}
{elapsed (totalcpu)}
{August 1989}
\machdesca{F-4/280}
{Sun 4/280}
{SunOS 4.0}
{Franz Allegro Common Lisp 3.0.1 beta}
{2x900}
{118MB}
\machdescb{32MB}
{Hi Res mono}
{Not running under GnuEmacs; no multi-processing; 4MB initial memory}
{}
{elapsed (user-run + system-run)}
{Winter 1989}
\machdesca{F-Sparc}
{Sun SparcStation}
{Sun OS 4.0}
{Franz Allegro Common Lisp 3.1.beta.20}
{600MB SCSI}
{Large}
\machdescb{16MB}
{Color}
{}
{}
{elapsed (totalcpu)}
{August 1989}
\machdesca{HP}
{Hewlett Packard 9000/350}
{Unix}
{HP Lisp 1.0}
{130MB (7958)}
{unknown}
\machdescb{16MB}
{color}
{under gnuemacs}
{}
{elapsed - run}
{Fall 1987}
\machdesca{K-3/75}
{Sun 3/75}
{SunOS 3.1}
{Kyoto Common Lisp ``September 16, 1986''}
{70MB SCSI}
{50MB local}
\machdescb{28MB}
{standard resolution mono}
{Files on Sun 3/180 NFS server}
{Under suntools}
{elapsed - run}
{Fall 1987}
\machdesca{L-3/75}
{Sun 3/75}
{SunOS 3.1}
{Lucid 2.0}
{70MB SCSI}
{50MB local}
\machdescb{28MB}
{standard resolution mono}
{Files on Sun 3/180 NFS server}
{used \pgm{:EXPAND 90 :GROWTH-RATE 90}}
{elapsed (user-run + system-run)}
{Fall 1987}
\machdesca{Mac2}
{Apple Macintosh II}
{Mac OS 5}
{Allegro Common Lisp 1.1}
{100MB internal}
{n/a}
\machdescb{5MB}
{E-machines Big Picture 17" mono}
{}
{}
{elapsed - paging}
{Spring 1988}
\machdesca{Maci}
{Symbolics MacIvory}
{Genera 7.4}
{Genera 7.4}
{300MB external}
{160,000kW (80MB)}
\machdescb{5,376kW (27MB); 2MB Mac II}
{Radius 19"}
{Apple EtherTalk}
{}
{elapsed - paging}
{April 24, 1989}
\machdesca{mX}
{Texas Instruments microExplorer}
{Explorer Lisp 5.0}
{Explorer Lisp 5.0}
{100MB Rodime}
{60MB}
\machdescb{12MB mX processor; 2MB Mac II}
{24" (1280 x 960) Moniterm Viking II}
{Apple EtherTalk}
{}
{elapsed - paging}
{December 1988}
\machdesca{NeXT}
{NeXT}
{0.9}
{Franz Allegro Common Lisp 3.1.0.5}
{Internal SCSI}
{unknown}
\machdescb{8MB}
{}
{}
{}
{elapsed (totalcpu)}
{August 1989}
\machdesca{RT}
{IBM RT/APC}
{AIX 2.1.2 (unix)}
{2.0.5 (Lucid 1.01)}
{``Fast'' EESDI controller; 3 x 70MB}
{80k x 512kB blocks (40,960MB)}
\machdescb{16MB of ``fast'' memory}
{Moniterm 1024 x 768 mono}
{AFT floating point unit; GSL windows}
{Used \pgm{:EXPAND 69} to get 6MB semispace;
This should be the fastest RT version now available}
{elapsed (user-run + system-run)}
{Spring 1988}
\machdesca{XCL}
{Xerox 1186}
{Xerox Lisp, Lyric release}
{Xerox Lisp, Lyric release}
{40MB}
{16MB}
\machdescb{3.5MB}
{19" mono}
{}
{}
{elapsed - gc - paging}
{Winter 1988}

\section{Test Procedures}
\label{testproc}

To run a BB1 test the following procedure was followed:
\begin{enumerate}
\item The \pgm{.LISP} files were copied to the host under test.

\item Lisp was restarted and any necessary configuration, such as disabling
end-of-screen processing, was done.

\begin{sloppypar}
\item If necessary, a \pgm{(PROCLAIM '(OPTIMIZE (SPEED~$x$) (SAFETY~$y$)))} form
was entered.
\end{sloppypar}

\item The form \pgm{(TIME (LOAD "COMPILE-BB1.LISP"))} was entered and allowed to
complete without interruption, and the resulting information was recorded.  A
side effect of loading the \pgm{COMPILE-BB1} file is that the source files are
compiled and loaded.

\item The form \pgm{(TIME (BB1::TEST-BBEDIT))} as entered and allowed to
complete.  The results were recorded.

\item Step 5 was repeated, which usually resulted in a better time.

\item Steps 2 through 6 were done a total of 4 times; once skipping step 3 and
then for $(X,�Y)�=�(3,�0)$, $(0,�3)$, and $(3,�2)$.

\end{enumerate}

To run a SOAR test the following procedure was followed:
\begin{enumerate}
\item The \pgm{.LISP} and \pgm{.SOAR} files were copied to the host under test.

\item Lisp was restarted and any necessary configuration, such as disabling
end-of-screen processing, was done.

\item The form \pgm{(TIME (COMPILE-FILE "SOAR.LISP"))} was entered and
allowed to complete.  The results were recorded.

\item Step 2 was repeated.

\item The following forms were entered:\pgm{\newline
\quad (LOAD "SOAR")\newline
\quad (LOAD "DEFAULT.SOAR")\newline
\quad (LOAD "EIGHT.SOAR")}

\item The form \pgm{(TIME (RUN-TASK))} was entered.
``1\fbox{\tt Return}\linebreak[0]3\fbox{\tt Return}'' was immediately typed
ahead as responses to the prompts soon to follow.  The timing was allowed to
complete and the results recorded for the \mode{A} mode test.

\item The form \pgm{(INIT-SOAR)} was entered.

\item Step 6 was repeated with
``1\fbox{\tt Return}\linebreak[0]1\fbox{\tt Return}'' in place of
``1\fbox{\tt Return}\linebreak[0]3\fbox{\tt Return}'' for the \mode{B} mode
test.

\item Steps 7 and 8 were repeated with
``3\fbox{\tt Return}\linebreak[0]3\fbox{\tt Return}'' in place of
``1\fbox{\tt Return}\linebreak[0]3\fbox{\tt Return}'' for the \mode{C} mode
test.

\item 	The following forms were entered:\newline\pgm{
\quad (EXCISE EIGHT*MONITOR-STATE)\newline
\quad (WATCH -1)}

\item Step 9 was repeated for the ``\mode{C} no trace'' mode test.

\item Steps 2 and 5 were repeated to reload SOAR.

\item Steps 7 and 8 were repeated for the ``\mode{A} no trace'' mode test.

\end{enumerate}

\end{document}
